In this work we have explored the impact of limitations imposed by physics in
physical reservoir computing (RC). Through Echo State Network (ESN) simulations
we have uncovered characteristics exhibited by white noise, quantization noise
and partially visible reservoirs in prediction of the NARMA10 time series.

By modeling additive white Gaussian noise (AWGN) we have found that the RC
paradigm provides an inherent robustness to noisy inputs. This is a promising
result when considering physical substrates as a reservoir medium, suggesting an
lower bound of 10dB SNR as tolerable to provide a satisfactory representation of
the signal in the reservoir dynamics.

Moreover, our experiments show that this resilience to white noise also extends
to ADC quantization noise. By quantizing the output of the $\tanh$ activation
function of hidden nodes in the ESNs, we have demonstrated that the general
reconstruction capability of our reservoirs is unhindered when using an ADC with
12 bits of resolution or more. This may be applied in preliminary analysis of
the resolution and representation of output activations.

We have also investigated the importance of input and output connectivity, as to
provide insights into reservoirs that are only partially visible to an
observer.

We found the optimal input connectivity to be when 10\% to 20\% of the reservoir
nodes see the input stream, and that the distribution of the input weights of
the ESNs is of little relevance to performance, which introduces flexibility in
the choice of reservoir perturbation.

No clear performance difference was found across reservoirs of varying sizes
when the amount of connected output nodes remains the same. This illustrates the
importance both of having a sufficient amount of output nodes, and that the
physical layout of these nodes should be such that enough of the reservoir
dynamics is captured.

The work presented contains an introductory exploration of physical limitations
in RC, and thus casts quite a wide net. Developing a concrete understanding of
the issues faced when moving simulation models to physical realizations is a
prerequisite to exploiting computational properties in physical systems. With
ESN simulations we have demonstrated the particular impact of several common
physical limits, which should be transferable to reservoirs manifesting similar
dynamics.

%%% Local Variables:
%%% mode: latex
%%% TeX-master: "../main"
%%% End:
