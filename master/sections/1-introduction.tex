\chapter{Introduction}
\label{ch:intro}

\textcolor{red}{
  Reservoir Computing spawned from difficulty of training RNNs. From Reservoir
Computing we now have a lot of physically based reservoirs. Introduce the
sentiment.
}

\textcolor{red}{
  Now, with physical reservoirs, we will have to adapt our previous Echo State
Networks to physical morphologies. Does this work?
}

\textcolor{red}{
  Spatial networks, can we use existing graph and lattice theory? Can we find
regular structures that exhibit the same characteristics we see in existing
models like ESN/LSM?
}

\textcolor{red}{
  Last two paragraphs should be: (i) we investigate two types of spatial
restrictions in \textit{simulation} to see what happens. And (ii), we use
spatial restriction models to propose new ways to do theoretical analysis when
we have spatial embedding.
}

\section{Research Goals}

In this thesis, we seek a better theoretical foundation for how reservoir
computing methodology translates to physical substrates. Specifically, we are
interested in understanding how spatially restricting the nodes of a reservoir
network impacts performance. Ultimately, the goal of the thesis is to answer the
following research questions:

\textbf{RQ1:} How does the reservoir computing paradigm translate to the
spatially constrained topology setting of a physical medium?

First, we are concerned with investigating practical challenges of realizing
physical reservoirs. We are interested in how restrictions, primarily embedding
reservoirs in physical space, will affect reservoir quality.

\textbf{RQ2:} How do highly regular, physical structures compare in information
processing capability to that of established models such as echo state networks?

Second, we are interested in how reservoirs with regular structures, such as
lattice grids, compare to established models. If there are discrepancies in
capability, we pursue the reasoning to gain a deeper theoretical insight into
why the regularity is disadvantageous.

\textbf{RQ3:} Can we find simple, deterministic reservoir generation
methodology, relying less on random weighting schemes?

Designing reservoirs in a highly deterministic manner is desirable, especially
due to the fact that it may simplify the physical realization process. Simple
schemes to embed nodes in space and establish connectivity are
beneficial. Additionally, less stochastic elements in reservoir generation may
allow us to peer into the ``black box'' character of reservoir computing.

\section{Thesis Overview}

The thesis is structured as follows. Chapter \ref{ch:intro} introduces the
research domain, and presents the motivation behind exploring spatial
constraints in reservoir computing. Chapter \ref{ch:background} covers relevant
background material used throughout the thesis. An overview of previous work on
physical reservoir computing is given in \ref{ssec:physreq}, and a discussion of
topology and spatial restrictions within reservoir computing is presented in
\ref{ssec:topology-and-spatial-networks}.

Chapter \ref{ch:method} describes the thesis methodology. Section
\ref{sec:esn-param} presents the parameters used for echo state network
generation, while Section \ref{sec:bench-metr} explains the implementation
details of benchmarks and other metrics.

Chapters \ref{ch:rgg} and \ref{ch:regular-tilings} present experiments made with
random geometric graphs and lattice networks, respectively. Most experiments
rely on the knowledge gained in previous sections, and thus follow a
chronological order.

Finally, Chapter \ref{ch:conclusion} summarizes the discoveries of the
experiments, draws conclusions related to the research goals of the thesis, and
suggests areas of future work.

%%% Local Variables:
%%% mode: latex
%%% TeX-master: "../thesis"
%%% End: