\chapter{Introduction}
\label{ch:intro}

\textcolor{red}{
  Reservoir Computing spawned from difficulty of training RNNs. From Reservoir
Computing we now have a lot of physically based reservoirs. Introduce the
sentiment.
}

\textcolor{red}{
  Now, with physical reservoirs, we will have to adapt our previous Echo State
Networks to physical morphologies. Does this work?
}

\textcolor{red}{
  Spatial networks, can we use existing graph and lattice theory? Can we find
regular structures that exhibit the same characteristics we see in existing
models like ESN/LSM?
}

\section{Research Goals}

Ultimately, the goal of the thesis is to answer the following research
questions.

\begin{itemize}
\item How does the Reservoir Computing paradigm translate to the spatially
constrained topology setting of a physical medium?
\item How do highly regular structures compare in terms of memory capacity,
information processing capacity, and robustness, to that of established models
such as echo state networks?
\item Can we find simple, deterministic weighting schemes for recurrent neural
networks, that are realizable as physical interactions?
\item Which characters/parameters in established models must necessarily
translate well to physical settings?
\end{itemize}

\textcolor{blue}{
  ``Reference [88] explores this idea by addressing three issues: (i) what is
the minimal complexity of topology and parameters that produce comparable
performance to standard models?  (ii) what degree of randomness is needed to
construct competitive reservoirs? and (iii) how do completely deterministic
reservoirs compare? These are good questions for understanding underlying RC
principles, but may be impractical to investigate given an already created
(maybe static) physical substrate.''
}

\section{Thesis Overview}

The thesis is structured as follows. Chapter \ref{ch:intro} introduces the
research domain, and presents the motivation behind exploring spatial
constraints in reservoir computing. Chapter \ref{ch:background} covers relevant
background material used throughout the thesis. An overview of previous work on
physical reservoir computing is given in \ref{ssec:physreq}, and a discussion of
topology and spatial restrictions within reservoir computing is presented in
\ref{ssec:topology-and-spatial-networks}.

Chapter \ref{ch:method} describes the thesis methodology. Section
\ref{sec:esn-param} presents the parameters used for echo state network
generation, while Section \ref{sec:bench-metr} explains the implementation
details of benchmarks and other metrics.

Chapters \ref{ch:rgg} and \ref{ch:regular-tilings} present experiments made with
random geometric graphs and lattice networks, respectively. Most experiments
rely on the knowledge gained in previous sections, and thus follow a
chronological order.

Finally, Chapter \ref{ch:conclusion} summarizes the discoveries of the
experiments, draws conclusions related to the research goals of the thesis, and
suggests areas of future work.

%%% Local Variables:
%%% mode: latex
%%% TeX-master: "../thesis"
%%% End: