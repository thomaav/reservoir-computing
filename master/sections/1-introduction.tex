\chapter{Introduction}
\label{ch:intro}

With the inevitable demise of Moore's law, researchers now seek computational
methods beyond the traditional transistor-based computer architecture. A wide
range of approaches, collectively dubbed \textit{unconventional computing}
methodology, aim to exploit the intrinsic computation present in many natural
systems. For example, \textit{evolution in-materio} shows that computation can
be implemented in physical systems as a hybrid of analogue and traditional
computation \cite{miller_evolution--materio_2014}.

Reservoir computing (RC) has become a predominant member of the unconventional
computing paradigm \cite{adamatzky_reservoir_2017}. It is a framework suited for
processing of temporal and sequential data, exploiting the underlying dynamics
of a \textit{reservoir}. Classical RC is derived from recurrent neural network
(RNN) models, e.g. echo state networks \cite{jaeger_echo_2001}. Utilizing an RNN
as a reservoir, input sequences are projected into a high-dimensional space,
incorporating its temporal information in an instantaneous readout. Training is
then carried out by adapting the readout layer with supervised linear
regression, providing faster and simpler training than traditional gradient
descent methods.

Interestingly, there is no need for the reservoir to be an artificial neural
network -- any high-dimensional, driven system exhibiting complex dynamic
behavior can be used \cite{schrauwen_overview_2007}. Through a fusion of
in-materio computation and the reservoir methodology, \textit{physical}
reservoir computing has seen a recent surge of interest
\cite{adamatzky_reservoir_2017, tanaka_recent_2018}.

In this work, the goal is to explore inherent limitations faced when exploiting
physical substrates as reservoirs. In the project preceding this thesis, noise,
equipment accuracy, and system observability were investigated as physical
limitations \cite{aven_exploring_2019}. Herein, we further this work, focusing
on constrained topology or physical morphology. Physical reservoirs must
necessarily be embedded in physical space, often having completely fixed
structural properties. This is in contrast to abstraction models -- physical
models are commonly limited in the way we may change its geometry. The role of
structure is a relatively unexplored area of RC.

\section{Research Goals}

In this thesis, we seek a better theoretical foundation for how reservoir
computing methodology translates to physical substrates. Specifically, we are
interested in understanding how spatially restricting the nodes of a reservoir
network impacts performance. Ultimately, the goal of the thesis is to answer the
following research questions:

\textbf{RQ1:} How does the reservoir computing paradigm translate to the
spatially constrained topology setting of a physical medium?

First, we are concerned with investigating practical challenges of realizing
physical reservoirs. We are interested in how restrictions, primarily embedding
reservoirs in physical space, will affect reservoir quality.

\textbf{RQ2:} How do highly regular, physical structures compare in information
processing capability to that of established models such as echo state networks?

Second, we are interested in how reservoirs with regular structures, such as
lattice grids, compare to established models. If there are discrepancies in
capability, we pursue the reasoning to gain a deeper theoretical insight into
why the regularity is disadvantageous.

\textbf{RQ3:} Can we find simple, deterministic reservoir generation
methodology, relying less on random weighting schemes?

Designing reservoirs in a highly deterministic manner is desirable, especially
due to the fact that it may simplify the physical realization process. Simple
schemes to embed nodes in space and establish connectivity are thus
beneficial. Additionally, less stochastic elements in reservoir generation may
allow us to peer into the ``black box'' character of reservoir computing.

To answer the posed research questions, we conduct simulations using traditional
echo state network methodology. We investigate two types of spatially
constrained network models as reservoirs: random geometric graphs and
lattices. Thus, since all experiments use echo state networks, we use a higher
level abstraction to model physical reservoirs by imposing specific structural
properties on the architecture. Generated networks are evaluated using widely
used approaches: the nonlinear autoregressive moving average (NARMA)
\cite{atiya_new_2000}, the Mackey-Glass delay differential equation
\cite{mackey_oscillation_1977}, kernel quality and generalization
\cite{legenstein_edge_2007}, and short-term memory \cite{jaeger_short_2002}.

\section{Thesis Overview}

The thesis is structured as follows. Chapter \ref{ch:intro} introduces the
research domain, and presents the motivation behind exploring spatial
constraints in reservoir computing. Chapter \ref{ch:background} covers relevant
background material used throughout the thesis. An overview of previous work on
physical reservoir computing is given in \ref{ssec:physreq}, and a discussion of
topology and spatial restrictions within reservoir computing is presented in
\ref{ssec:topology-and-spatial-networks}.

Chapter \ref{ch:method} describes the thesis methodology. Section
\ref{sec:esn-param} presents the parameters used for echo state network
generation, while Section \ref{sec:bench-metr} explains the implementation
details of benchmarks and other metrics.

Chapters \ref{ch:rgg} and \ref{ch:regular-tilings} present experiments made with
random geometric graphs and lattice networks, respectively. Most experiments
rely on the knowledge gained in previous sections, and thus follow a
chronological order.

Finally, Chapter \ref{ch:conclusion} summarizes the discoveries of the
experiments, draws conclusions related to the research goals of the thesis, and
suggests areas of future work.

%%% Local Variables:
%%% mode: latex
%%% TeX-master: "../thesis"
%%% End: