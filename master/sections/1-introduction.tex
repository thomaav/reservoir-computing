\chapter{Introduction}

Reservoir Computing spawned from difficulty of training RNNs. From Reservoir
Computing we now have a lot of physically based reservoirs. Introduce the
sentiment.

Now, with physical reservoirs, we will have to adapt our previous Echo State
Networks to physical morphologies. Does this work?

Spatial networks, can we use existing graph and lattice theory? Can we find
regular structures that exhibit the same characteristics we see in existing
models like ESN/LSM?

\section{Goals}

Research questions:

\begin{itemize}
\item How does the Reservoir Computing paradigm translate to the spatially
constrained topology setting of a physical medium?
\item How do highly regular structures compare in terms of memory capacity,
information processing capacity, and robustness, to that of established models
such as echo state networks?
\item Can we find simple, deterministic weighting schemes for recurrent neural
networks, that are realizable as physical interactions?
\item Which characters/parameters in established models must necessarily
translate well to physical settings?
\end{itemize}

``Reference [88] explores this idea by addressing three issues: (i) what is the
minimal complexity of topology and parameters that produce comparable
performance to standard models?  (ii) what degree of randomness is needed to
construct competitive reservoirs? and (iii) how do completely deterministic
reservoirs compare? These are good questions for understanding underlying RC
principles, but may be impractical to investigate given an already created
(maybe static) physical substrate.''

\section{Overview}

Chapters are about so and so.

%%% Local Variables:
%%% mode: latex
%%% TeX-master: "../thesis"
%%% End: