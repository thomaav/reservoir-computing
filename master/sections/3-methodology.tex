\chapter{Methodology}

\textcolor{red}{
  Anyway: for each experiment, describe the experimental setup that was used
such that it may \textit{easily be reproduced}. It has been very helpful when
papers actually provide every relevant parameter such that I could run the
experiments myself.
}

\textcolor{red}{
  We haven't done much different from the pre-thesis (METHODS chapter), so the
methodology approach from there can probably adapted to this. But we also need
to introduce methodology for kernel quality, generalization and memory capacity,
as the pre-thesis only used NARMA-10.
}

\textcolor{red}{
  We do not really do much more than replace the reservoir, everything else
remains much the same.
}

\section{ESN Parameters and Sample Sizes}

\textcolor{red}{
  Introduce the baseline Echo State Network that we used, and what types of
training data we used for offline training and so on.
}

\textcolor{red}{
  Just repeating runs vs. cross validation. Common approach is just doing a
bunch of runs.
}

\textcolor{red}{
  A parameter is Ridge Regression with an SVD solver for computational
routines. We also tried Cholesky solver and pseudo inverse.
}

\section{Benchmarks and Metrics}

\textcolor{red}{
  NARMA-10. Implementation of KQ/G and MC as well here. Show memory curve thing,
how much per delay is remembered.
}

\textcolor{red}{
  The Python software library implementation is available
online\footnote{\textcolor{red}{deadlink}}.
}

%%% Local Variables:
%%% mode: latex
%%% TeX-master: "../thesis"
%%% End: