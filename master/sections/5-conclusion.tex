\chapter{Conclusion and Future Work}
\label{ch:conclusion}

\section{Conclusion}

In this thesis we have explored physical reservoir computing with spatial
constraints. Our goal was to provide a better theoretical foundation for how
reservoir computing methodology translates to physical substrates. We conducted
software simulations using echo state network methodology to investigate the
feasibility of reservoirs with two different spatial restrictions: random
geometric graphs and lattices. Furthermore, we have developed lattice reservoirs
that can be used for theoretical analysis of ESN internals, accompanied with
example analyses of deterministic reservoir construction.

\textbf{RQ1:} How does the reservoir computing paradigm translate to the
spatially constrained topology setting of a physical medium?

The difference from abstraction models such as ESNs to physical reservoirs is
primarily concerned with the \textit{limitations} posed when interacting with an
actual physical substrate. In this work we have investigated spatial
limitations, and in Section \ref{sec:rwd} we emphasized that the weight
distribution resulting from a spatially constrained reservoir is different from
unrestricted architectures. The translation of the RC paradigm to a spatial,
physical setting is thus primarily limited by the (possibly fixed) geometries of
the underlying substrate how its internal units are connected. Other limitations
than spatial constraints, such as noise and system observability, were explored
in preliminary work \cite{aven_exploring_2019}.

More specifically, we have shown that ESNs with imposed spatial limitations by
default see a decrease in performance compared to their abstract
counterparts. However, both RGG and lattice reservoirs achieved improved
performance once the symmetry in the resulting internal reservoir matrix was
broken by directed and signed edges.

\textbf{RQ2:} How do highly regular, physical structures compare in information
processing capability to that of established models such as echo state networks?

\textcolor{red}{
  Highly regular structures investigated: lattices. We see something similar to
Rodan and Tino.
}

\textbf{RQ3:} Can we find simple, deterministic reservoir generation
methodology, relying less on random weighting schemes?

\textcolor{red}{
  We have introduced methodology to do so, since the directedness is the
stochastic element we can use this methodology to further our understanding,
which is also the intention of CRJ.
}

\textcolor{red}{
  Some thoughts on what we have discovered: (1) Removal of nodes can be a
convenient measure (perhaps something similar to ``Reservoir computing: Reducing
network size and improving prediction stability'', but I think maybe this paper
makes an erroneous assumption?). (2) Directedness by itself is
\textit{sufficient} by itself, but does not necessarily imply that it is a
textit{must}. (3) A method of visualizing the internals of reservoirs with some
degree of randomness, as opposed to CRJ, but easier to look at than ESNs (black
box problem). (4) Introduce a notion that there might be some core ``stem'' in
ESNs that correspond to a memory, that is operated on by augmentative nodes.
}

\section{Future Work}

\textcolor{red}{
  What should be done? What \textit{would} we have done if we were to continue
investigating?
}

\textcolor{red}{
  real life may not be in the same lockstep, so real life experiments are
  interesting. for example having directedness in artificial spin ice.
}

%%% Local Variables:
%%% mode: latex
%%% TeX-master: "../thesis"
%%% End: