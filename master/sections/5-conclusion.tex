\chapter{Conclusion and Future Work}
\label{ch:conclusion}

\section{Conclusion}

In this thesis we have explored physical reservoir computing with spatial
constraints. Our goal was to provide a better theoretical foundation for how
reservoir computing methodology translates to physical substrates. We conducted
software simulations using echo state network methodology to investigate the
feasibility of reservoirs with two different spatial restrictions: random
geometric graphs and lattices. Furthermore, we have developed lattice reservoirs
that can be used for theoretical analysis of ESN internals, accompanied with
example analyses of deterministic reservoir construction.

\textbf{RQ1:} How does the reservoir computing paradigm translate to the
spatially constrained topology setting of a physical medium?

The difference from abstraction models such as ESNs to physical reservoirs is
primarily concerned with the \textit{limitations} posed when interacting with an
actual physical substrate. In this work we have investigated spatial
limitations, and in Section \ref{sec:rwd} we emphasized that the weight
distribution resulting from a spatially constrained reservoir is different from
unrestricted architectures. The translation of the RC paradigm to a spatial,
physical setting is thus primarily limited by the (possibly fixed) geometries of
the underlying substrate how its internal units are connected. Other limitations
than spatial constraints, such as noise and system observability, were explored
in preliminary work \cite{aven_exploring_2019}.

More specifically, we have shown that ESNs with imposed spatial limitations by
default see a decrease in performance compared to their abstract
counterparts. However, both RGG and lattice reservoirs achieved improved
performance once the symmetry in the resulting internal reservoir matrix was
broken by directed and signed edges.

\textbf{RQ2:} How do highly regular, physical structures compare in information
processing capability to that of established models such as echo state networks?

Highly regular lattice structures with \textit{bidirectional} edges perform
worse than ESNs. However, our investigations also demonstrated that introducing
directedness to lattice reservoirs restores the ESN performance, indicating that
structures that allow a definitive flow of information may be sufficient. We
also showed that a fixed, global input betters performance compared to a
standard uniform distribution, making lattice reservoirs scale better than
traditional ESNs on the NARMA-10 benchmark. We thus observe that regular
structures may perform just as well as established models, as both previous work
on ring topology as well as the work in this thesis suggests that
deterministically constructing regular reservoirs has revealed great potential.

\textbf{RQ3:} Can we find simple, deterministic reservoir generation
methodology, relying less on random weighting schemes?

It is previously established that ring topology models may be constructed
deterministically to serve as quality reservoirs \cite{rodan_minimum_2011}. In
this thesis we introduced lattice reservoir models, which are constructed by
deterministically placing nodes on a two-dimensional grid. Both ring and lattice
reservoirs, however, require some stochastic element to work well. In the case
of rings the sign of the input seen by each node is determined by an unbiased
coin, while in the case of lattices it is the direction of the edge between each
node that is decided by a coin flip. We therefore argue that there is untapped
potential in creating reservoirs of deterministic, regular natures.

Finally, by removing and adding nodes to lattice reservoirs, we illustrated
methodology for theoretical analysis of the inner workings of such
networks. Firstly, we found that a removal of a few nodes can be a convenient
measure to reduce network size and improve prediction performance. More
importantly, we also argue that understanding the behavior of networks when
solving specific benchmarks is a stepping stone to looking into the general
black box behavior of ESNs, and that the regular structure of lattice reservoirs
is easier to observe directly than the more stochastic structure of ESNs. Hence,
the intention of lattice reservoirs is not to provide yet another ESN
competitor, but to open up for further analysis in future work.

\section{Future Work}

As there is always an abundance of future work to conduct, we here limit
suggestions to two main categories of experiments we deem the most interesting:
physical realization of reservoirs to determine if directed exchange and flow of
information is sufficient to create quality reservoirs, and deeper theoretical
analysis using lattice models to find interesting heuristics in good reservoirs.

First, as ASI has proven to be a promising substrate for reservoir computing, it
is a good candidate for study of real physical limitations
\cite{jensen_reservoir_2020}. In this thesis it is suggested that directedness
is a key component in good reservoirs, and how this translates to an actual
physical medium would be interesting to explore. For example, imposing shift
register structures onto ASI to allow for directed flows may hold a potential
for creating quality reservoirs.

Second, further theoretical analysis using lattice models should result in a
deeper understanding of why the traditional, stochastic ESN model works so
well. By modifying lattice models to find good heuristics such as average node
degree, investigate cyclic structures, add skip edges, and so on, we may
discover what it is that makes ESN reservoirs ``tick''. Specifically, lattice
reservoirs are inherently embedded in space, making them easier to visualize and
understand. This is valuable when attempting to understand how specific tasks
are solved by an ESN.

%%% Local Variables:
%%% mode: latex
%%% TeX-master: "../thesis"
%%% End: