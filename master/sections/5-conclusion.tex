\chapter{Conclusion}

\section{Conclusion}

\textcolor{red}{
  What was achieved? Remember to \textit{very specifically} answer the research
questions presented in the introduction. What did we set out to achieve, and
what was concretely achieved.
}

\textcolor{red}{
  Remember to include shortcomings here (which should also be discussed in the
discussion section, of course).
}

\textcolor{red}{
  Some thoughts on what we have discovered: (1) Removal of nodes can be a
convenient measure (perhaps something similar to ``Reservoir computing: Reducing
network size and improving prediction stability'', but I think maybe this paper
makes an erroneous assumption?). (2) Directedness by itself is
\textit{sufficient} by itself, but does not necessarily imply that it is a
textit{must}. (3) A method of visualizing the internals of reservoirs with some
degree of randomness, as opposed to CRJ, but easier to look at than ESNs (black
box problem). (4) Introduce a notion that there might be some core ``stem'' in
ESNs that correspond to a memory, that is operated on by augmentative nodes.
}

\textcolor{blue}{
  ``This is somewhat counterin-tuitive as large weights in the output function
suggest astrong influence of the respective node in the aggregationof the
(correct) output signal.'' This is not necessarily true, as we saw in some of
our experiments. ``Our results demonstrate that a large optimization po-tential
lies in a systematical refinement of the differen-tial reservoir properties for
a given data set.  This wasoutlined on the examples of controlled node removal
andintroduction of a scaling factor in the activation function.'' Also quite
relevant.
}

\section{Future Work}

\textcolor{red}{
  What should be done? What \textit{would} we have done if we were to continue
investigating?
}

\textcolor{red}{
  If one wants to determine if square grids is comparable to ESNs, more
benchmarks are needed. We just studied whether they are usable at all.
}

%%% Local Variables:
%%% mode: latex
%%% TeX-master: "../thesis"
%%% End: