Further investigations is inclined to be of twofold nature. First, a deeper
exploration of physical limitations should include physical implementations of
reservoir systems to compare their performance tendencies to simulated
systems. A correlation between the performance of the simulation and the
physical realization will tell us something about the generality of the
limitations we study. Next, once a specific physical system is chosen, the
specific limitations that are relevant for that system should be studied.

Artificial spin ice computation has proven to be a promising substrate for
material computation \cite{jensen_computation_2018}, making it a good candidate
for a study of real physical limitations.

Arranging a large number of nanomagnets in a 2D lattice presents topological
limitations that is not naturally present in ESN simulations. An interesting
topic is thus how enforcing the layout of the system to be two-dimensional
affects computational capabilities. The physical morphology of the lattice ,
e.g. monoclinic, hexagonal and tetragonal crystal families, may play interesting
roles for computation. Imposing such restrictions in simulation may reveal
insights about physical systems.

Another restriction in artificial spin ice systems is that of a distinction
between the microstate and the macrostate. Studying the magnetic spin of
individual particles is a computationally infeasible endeavor, hence requiring
implementations to observe the system in some macrostate manner. The
degeneration of many microstates to one macrostate, whether by averaging or
otherwiser, could alter the observed dynamics.

Finally we would like to point out a particularly compelling limitation when
considering substrates that do not stem from biological evolution. In this work
we show that the ESN model is robust to input errors such as noise, but are not
concerned with the effect of internal model errors. Although self-organizing,
biologically inspired systems show an inherent robustness to defects, e.g. dead
or noisy cells in a neural networks, this may not be the case in other
substrates. We believe internal malfunctions to be a restriction that could
manifest itself in physical reservoirs.

%%% Local Variables:
%%% mode: latex
%%% TeX-master: "../main"
%%% End:
