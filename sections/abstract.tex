<\begin{abstract}
  Reservoir Computing (RC) emerged as an alternative framework to the
traditional gradient descent methods for training Recurrent Neural Networks
(RNNs). Key to the RC methodology is a randomly generated reservoir, commonly an
RNN that remains untrained, and a linear readout layer that is trained using
simple one-shot learning methods. Interestingly, there is no need for the
reservoir to be an artificial neural network -- any high-dimensional, driven
system exhibiting complex dynamic behavior can be used. A wide range of physical
reservoirs have been realized, ranging from optical laser circuits and
nanomagnetic assemblies to biological neural networks. The computational
performance of such physical substrates is closely related to common physical
limitations, e.g. noise, measurement accuracy, partially visible reservoir
state, and physical morphology. Here we investigate these fundamental properties
of physical reservoirs, offering insights into impairments that may be present
with such limitations, which in the wider context help improve the design of
future substrates.

% (TODO): concrete examples that help improve the design.

\end{abstract}

\begin{IEEEkeywords}
  Reservoir computing, unconventional computing, echo state networks, time
series computing.

  % (TODO): finish this list.

\end{IEEEkeywords}

%%% Local Variables:
%%% mode: latex
%%% TeX-master: "../main"
%%% End: