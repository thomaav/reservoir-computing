\begin{abstract}
  Reservoir Computing (RC) emerged as an alternative framework to the
traditional gradient descent methods for training Recurrent Neural Networks
(RNNs). Interestingly, there is no need for the reservoir to be an artificial
neural network, and in recent years a wide range of physical reservoirs have
been realized, ranging from optical laser circuits and nanomagnetic assemblies
to biological neural networks. The computational performance of such physical
substrates is likely to be closely related to common physical limitations,
e.g. noise, measurement accuracy, partially visible reservoir state, and
physical morphology. Here we investigate fundamental properties of physical
reservoirs using Echo State Network (ESN) simulations, offering insights into
impairments resulting from such limitations, which in the wider context will
help improve the design of further substrates.

We find reservoirs to be very robust to noisy inputs. Furthermore, we observe
that ADCs with 12 bits of resolution are sufficient to represent reservoir
dynamics well. Finally, we suggest approaches to reservoir perturbation and
observation, most importantly discovering that information may be completely
lost with output sampling that is too sparse.
\end{abstract}

\begin{IEEEkeywords}
  Reservoir computing, unconventional computing, echo state networks, time
series computing, physical limits, noise.
\end{IEEEkeywords}

%%% Local Variables:
%%% mode: latex
%%% TeX-master: "../main"
%%% End: