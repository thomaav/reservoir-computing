In this paper, we consider discrete-time ESNs with $N$ internal network nodes, a
single input, and a single output node.

The default reservoir size used is 200 hidden nodes. $\mathbf{W}^{res}$ and
$\mathbf{W}^{in}$ are both generated as random matrices with i.i.d. entries in
the interval [-0.5, 0.5]. Both matrices are fully connected, and the reservoir
weight matrix was rescaled such that $\rho(\mathbf{W}_{res}) = 0.9$.  Input
scaling is set to $\iota = 1.0$. $\mathbf{W}_{out}$ is adapted with the
Moore-Penrose pseudo-inverse using singular value decomposition, as this was
found to lead to the best results.

The first 200 states of each run are discarded to provide a washout of the
initial reservoir state. For all experiments, the generated input was split into
a training and test set, with $L_{train} = 2000$ and $L_{test} = 3000$. Reported
performances are the mean across ten randomizations of each model
representative.

All reservoirs were constructed with the parameters from this baseline,
differing only in the parameters given for each specific experiment. The Python
software library implementation is available online\footnote{Some GitHub
repository.}.

\subsection{Noise}

We model AWGN by extending the ESN model to take the sum of two individual
inputs, $\mathbf{u}(t)$ and $\mathbf{v}(t)$, which represent the signal and the
noise. The goal of the reservoir remains a computation on the signal
$\mathbf{u}(t)$, a task now hindered by the unwanted noise.

We vary the signal to noise ratio of the injected noise when running the test
dataset. The signal to noise ratio is measured in dB, and is calculated as $SNR
= 10\log_{10}(\frac{var(u)}{var(v)})$.

\subsection{Measurement equipment accuracy}

To emulate the behavior of an ADC, we extend our ESN model to allow for
quantization of reservoir output before it is passed to the readout layer. This
quantization effectively divides the range of the nonlinear activation function
of each hidden node into a discrete set of fixed output bins.

Fig. \ref{adc_quantization} shows how quantization affects reservoir
performance. We plot the error of four different reservoir sizes: 50, 100, 200
and 400 hidden nodes to see whether it is possible to compensate for lower
resolutions by increasing the size of the reservoir.

\subsection{Partially visible reservoir state}

We begin by experimenting with the sparsity of $\mathbf{W}_{in}$ and
$\mathbf{W}_{out}$. In both cases, we now generate the connection matrices such
that a wanted density, given as the fraction of connected nodes, is
achieved. Input and output is adjusted separately. Fig. \ref{partial_visibility}
shows the results of our simulation runs.

Additionally, we examine three different input weight distributions: uniform,
Gaussian and fixed. All inputs are all sampled as i.i.d streams. The uniform
distribution is sampled in the interval [-0.5, 0.5], the Gaussian distribution
is sampled with a zero mean and standard deviation $\sigma = 1.0$, and the fixed
distribution has every input weight set to 1. Moreover, we explore the parameter
space of the input scaling in the interval [0.1, 1.0].

%%% Local Variables:
%%% mode: latex
%%% TeX-master: "../main"
%%% End:
