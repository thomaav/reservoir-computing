Training Recurrent Neural Networks (RNNs) is an inherently difficult task
\cite{bengio_learning_1994}. To combat the high algorithmic complexity of
previous training methods, Echo State Networks (ESNs) present an alternative
supervised learning technique that does not adapt the internal weights of the
network \cite{jaeger_echo_2001}. Instead, the output is generated using a
simple, memoryless classifier or regressor, making the function of the internal
RNN resemble that of kernel methods. Thus, by projecting the input sequence into
a high-dimensional space, the temporal information of a time series may be
incorporated in the instantaneous readout. This methodology, concerned with
exploiting the underlying dynamics of a \textit{reservoir}, is unified in the
research subfield of Reservoir Computing (RC).

Interestingly, there is no need for the reservoir to be an artificial neural
network -- any high-dimensional, driven system exhibiting complex dynamic
behavior can be used \cite{schrauwen_overview_2007}. Reservoirs are thus
designed such that we are able to harness the dynamics that governs the
\textit{substrate} that implements it. A multitude of substrates have shown
promise as reservoirs: electronic memristor circuits
\cite{kulkarni_memristor-based_2012}, photonic systems
\cite{vandoorne_experimental_2014}, mechanical springs
\cite{hauser_towards_2011}, and more biologically oriented reservoirs such as
gene regulation networks \cite{jones_is_2007}, and the cat primary visual cortex
\cite{scholkopf_temporal_2007}. Consult \cite{tanaka_recent_2018} for an
overview of recent advances in physical RC.

In this paper we seek to investigate fundamental properties related to physical
reservoirs. Commonly, preliminary studies are conducted by simulating the
proposed dynamical system numerically to gauge its applicability in a physical
reservoir setting. Such models may not entirely encapsulate the uncertainties
and limitations that will exist in its corresponding physical setting, thus
leading to a divergence between the performances observed
\cite{vandoorne_experimental_2014, katumba_neuromorphic_2018,
  jensen_reservoir_2017}.

The extent of the performance degradation caused by physical limitations is not
readily understood. Hence, as implementations of reservoir systems increasingly
tend toward physical substrates, the computational performance may be affected
by intrinsic physical limitations. We explore noise, measurement equipment
accuracy, and partially visible reservoir state.

Reservoir robustness to noise is a primary concern that has been demonstrated
both numerically and experimentally in an optoeletronic setting, where
pre-processing techniques have been shown to reduce performance degradations
\cite{soriano_optoelectronic_2013}. It is well-established that ESNs are
resilient to internal noise \cite{jaeger_echo_2001}, but it is uncertain whether
this translates to intrinsically noisy inputs. ESNs provide a natural context
for studying the general significance of noise.

Another relevant noise characteristic is that of equipment
accuracy. Quantization noise, i.e. the resolution of the interface equipment, is
usually determined by DAC and ADC instruments. Employing equipment with
sufficient resolution has been found to be important when using physical
reservoirs \cite{soriano_delay-based_2015}.

Physical substrates will also differ in their process for both input
perturbation and state observation. The impact of input and output density,
i.e. the amount of reservoir nodes that are accessible to an observer, will
impact performance.

Lastly, being able to vary the input magnitude that each node sees freely is
desirable. Scaling the input signal before it is presented to the network is
usually easily accomplished, but applying individual weights for each internal
node is not, especially when considering a restrictive topology. Simple
weighting schemes, such as fixing all input weights to the same value are of
interest.

The rest of this paper is structured as follows: first we provide relevant
background theory, including the motivation behind the exploration of each
physical limitation. Next we present methods and the simulation setup, followed
by a section on results and discussion. Finally we draw conclusions and suggest
future work.

%%% Local Variables:
%%% mode: latex
%%% TeX-master: "../main"
%%% End: