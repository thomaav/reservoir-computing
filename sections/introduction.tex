Training Recurrent Neural Networks (RNNs) is an inherently difficult task
\cite{bengio_learning_1994}. To combat the high algorithmic complexity of
previous training methods, Echo State Networks (ESNs) \cite{jaeger_echo_2001}
present an alternative supervised learning technique that does not adapt the
internal weights of the network. Instead, the output is generated using a
simple, memoryless classifier or regressor, making the function of the internal
RNN resemble that of kernel methods. Thus, by projecting the input sequence into
a high-dimensional space, the temporal information of a time series may be
incorporated in the instantaneous readout. This methodology, concerned with
exploiting the underlying dynamics of a \textit{reservoir}, is unified in the
research subfield of Reservoir Computing (RC).

Interestingly, there is no need for the reservoir to be an artificial neural
network -- any high-dimensional, driven system exhibiting complex dynamic
behavior can be used \cite{schrauwen_overview_2007}. Reservoirs are thus
designed such that we are able to harness the dynamics that governs the
\textit{substrate} that implements it. A multitude of substrates have shown
promise as reservoirs: electronic memristor circuits
\cite{kulkarni_memristor-based_2012}, photonic systems
\cite{vandoorne_experimental_2014}, mechanical springs
\cite{hauser_towards_2011} and more biologically oriented reservoirs such as
gene regulation networks \cite{jones_is_2007} and the cat primary visual cortex
\cite{scholkopf_temporal_2007}. Consult \cite{tanaka_recent_2018} for an
overview of recent advances in physical RC.

Commonly, preliminary studies are conducted by simulating the proposed dynamical
system numerically to gauge its applicability in a physical reservoir
setting. Such models may not entirely encapsulate the uncertainties and
limitations that will exist in its corresponding physical setting, thus leading
to a divergence between the performances observed
\cite{vandoorne_experimental_2014, katumba_neuromorphic_2018,
jensen_reservoir_2017}. The extent of performance degradation caused by physical
limitations is not readily understood, and presents a knowledge gap in the field
of physical RC.

Hence, as implementations of reservoir systems increasingly tend toward physical
substrates, the computational performance may be affected by intrinsic physical
limitations. In this paper we seek to investigate fundamental properties related
to physical reservoirs: how does noise affect performance? How does the accuracy
of our measurement equipment affect reservoir quality? What if we are only able
to partly observe the reservoir?

The rest of this paper is structured as follows:

% (TODO): Structure.

% (TODO): Add information about Chua's diode as example physical implementation.

% (TODO): Physical substrates present difficulties, no longer possible to just
% tune the spectral radius (keys from practical guide: spectral radius, input
% scaling, leakiness).

%%% Local Variables:
%%% mode: latex
%%% TeX-master: "../main"
%%% End: