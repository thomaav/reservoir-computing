Training Recurrent Neural Networks is an intrinsically difficult task, as
gradient descent methods that use loss information become increasingly
inefficient on problems with long-range temporal dependencies (TODO: as the
range grows, rewrite).  A continuous search in the parameter space may cause
bifurcations points in the dynamics of the system, causing non-convergent
training \cite{doya_bifurcations_nodate}. Moreover, due to the nature of
vanishing gradients, a converging network is slow and expensive to train, and
will oftentimes lead to insufficient local minima (TODO: change vanishing
gradients to something better, and rewrite this to fit citation)
\cite{bengio_learning_1994}. Next: LSM and ESN introduction, RC introduction.

While error backpropagation Recurrent Reservoir Computing (RC) but forth as a
way to compute with recurrent neural networks.

Readout is memoryless.

RC may have left being RNN training method, to way of using any substrate
computation.

RNNs are hard. Bifurcations. Kenji Doya's paper.

Physical substrates present difficulties, no longer possible to just tune the
spectral radius.

LSM. ESN. Also spawned physical reservoirs: neurons, echoli, lasers etc. Consult
Tanaka for full list.

Readout typically linear feed forward, can be written as y = wx etc. Ridge
regression and other methods.

Evaluation of ESN performance on NARMA10 has been done in: An experimental unification of reservoir computing methods, Minimum Complexity Echo State Networks, Adaptive Nonlinear System Identification with Echo State Networks

''Natural computational systems must have specific topologies, and the uniform
random connectivity is not appropriate.'' Topology discussion, Manevitz et. al.

%%% Local Variables:
%%% mode: latex
%%% TeX-master: "../main"
%%% End: