Evaluation of ESN performance on the NARMA system is a thoroughly explored area
in the field of RC \cite{verstraeten_experimental_2007, rodan_minimum_2011,
jaeger_adaptive_nodate}. Similar performance to previous work has been achieved
(Fig. \ref{visualization}, \ref{performance}) as a baseline to lend credibility
to further approaches in this paper.

\begin{figure}[htbp]
  \centering
  \includegraphics[width=2.5in]{img/narma_visualization.png}
  \caption{
    Visualization of reservoir output when fed the NARMA10 task. Input is an
i.i.d. stream generated uniformly in the interval [0, 0.5].
  }
  \label{visualization}
\end{figure}

\begin{figure}[htbp]
  \centering
  \includegraphics[width=2.5in]{img/general_performance.png}
  \caption{
    Reservoir size versus ESN performance for the NARMA10 task. The NRMSE is
averaged over 10 simulation runs.
  }
  \label{performance}
\end{figure}

All further reservoirs were constructed with the parameters from this baseline:
$\mathbf{W}^{res}$ and $\mathbf{W}^{in}$ were both generated as random matrices
with i.i.d. entries in the interval [-0.5, 0.5]. Both matrices are fully
connected, and the reservoir weight matrix was rescaled to have a spectral
radius of 0.9.

\subsection{Noise}

\subsection{Measurement equipment accuracy}

\subsection{Partially visible state}

\begin{figure*}[htbp]
  \centering
  \begin{subfigure}{.3\textwidth}
    \centering
    \includegraphics[width=\linewidth]{img/input_density_all.png}
  \end{subfigure}
  \begin{subfigure}{.3\textwidth}
    \centering
    \includegraphics[width=\linewidth]{img/output_density_all.png}
  \end{subfigure}
  \begin{subfigure}{.3\textwidth}
    \centering
    \includegraphics[width=\linewidth]{img/partial_visibility.png}
  \end{subfigure}
  \caption{
    These figures show the influence of reservoirs that are only partially
visible on the performance for the NARMA10 task. Experiments for the rightmost
plot were conducted using a reservoir size of 200 hidden nodes. The density is a
measurement for the fraction of elements in the input and output matrices
containing non-zero elements.
  }
  \label{performance}
\end{figure*}

\begin{figure}[htbp]
  \centering
  \includegraphics[width=2.5in]{img/input_scaling_distrib.png}
  \caption{
    Effect of input scaling on three input weight distributions.
  }
  \label{performance}
\end{figure}

% (TODO): What about other weight distributions?

% (TODO): Is w_density also interesting for physical reservoirs?

\subsection{Topology}

%%% Local Variables:
%%% mode: latex
%%% TeX-master: "../main"
%%% End:
