Evaluation of ESN performance on the NARMA system is a thoroughly explored area
in the field of RC \cite{rodan_minimum_2011, verstraeten_experimental_2007,
jaeger_adaptive_nodate}. Similar performance to previous work has been achieved
(Fig. \ref{visualization}, \ref{performance}) as a baseline to lend credibility
to further approaches in this paper.

\begin{figure}[H]
  \centering
  \includegraphics[width=2.5in]{img/narma_visualization.png}
  \caption{
    Visualization of reservoir output when fed the NARMA10 task. Input is an
i.i.d. stream generated uniformly in the interval [0, 0.5].
  }
  \label{visualization}
\end{figure}

\begin{figure}[H]
  \centering
  \includegraphics[width=2.5in]{img/general_performance.png}
  \caption{
    Reservoir size versus ESN performance for the NARMA10 task. The NRMSE is
averaged over 10 simulation runs.
  }
  \label{performance}
\end{figure}

All further reservoirs were constructed with the parameters from this baseline,
unless otherwise specified. The default reservoir size used was 200 hidden
nodes. $\mathbf{W}^{res}$ and $\mathbf{W}^{in}$ were both generated as random
matrices with i.i.d. entries in the interval [-0.5, 0.5]. Both matrices are
fully connected, and the reservoir weight matrix was rescaled such that
$\rho(\mathbf{W}_{res}) = 0.9$. The first 200 states of each run are discarded
to provide a \textit{washout} of the initial reservoir state. For all
experiments, the generated input was split into a training and test set, with
$L_{train} = 2000$ and $L_{test} = 3000$. All reported performances are the mean
across ten randomizations of each model representative. The Python software
library implementation is available online\footnote{Some GitHub repository.}.

% (TODO): Split this entire section into one on methodology and one on discussion,
% such that we first introduce just the bare minimum of what has been done, and
% then actually discuss it coherently later.

\subsection{Noise}

% (TODO): Write about main motivation for doing this: what if we have a model
% e.g. in software that we are now implementing physically: how does the model
% then respond to the noise?

% (TODO): Find some more background information on previous works, both in general
% and with ESN in particular. What is the contribution of this work at all?

Physical, real world systems are affected by noise. By extension, designers of
reservoirs that use material substrates must be aware of the effects the noise
that is present may have on computational power.

It is well known in the field of traditional artificial neural networks that an
addition of noise to input data can lead to generalization improvements similar
to that of Tikhonov Regularization \cite{bishop_training_1995}. This has been
verified to hold for the RC paradigm \cite{kurkova_stable_2008}, citing that the
more pragmatic approach is simply using ridge regression, as dynamic noise
injection is a non-deterministic approach. In this section, we investigate the
impact of adding noise to just the test set, and hence exploring reservoir
robustness to noisy environments.

Additive white Gaussian noise (AWGN) is a common noise model that mimics the
noise patterns of many random processes in nature. The noise is additive,
meaning the AWGN output is the sum of the input $u_{i}$ and the noise values
$v_{i}$. $v_{i}$ is i.i.d and drawn from a Gaussian distribution with zero-mean,
and a variance $\sigma^{2}$.

We model AWGN noise by extending the ESN model to take the sum of two individual
inputs, $u$ and $v$, which represent the signal and the noise. The goal of the
reservoir remains a computation on the signal $u$, a task now hindered by the
unwanted noise.

We vary the signal to noise ratio of the injected noise when running the test
dataset. Thus we will see the inherent robustness of the reservoirs to undesired
noise that is previously unseen. The results are shown in
Fig. \ref{input_noise_snr}, illustrating a slight performance degradation when
the ratio of signal power to noise power drops below 20 dB. The reservoir
performance drops more drastically when reaching 10 dB. Similar performance
degradation was seen in \cite{dambre_information_2012}, where the same SNR
measure was used to evaluate the signal reconstruction capacity from the state
of a dynamical system.

\begin{figure}
  \centering
  \includegraphics[width=2.5in]{img/input_noise_snr.png}
  \caption{
    Noise causing a decrease in performance of a reservoir with 200 hidden nodes
on the NARMA10 task. Input signal to noise ratio is measured in dB, and is
calculated as $SNR = 10\log_{10}(\frac{var(u)}{var(v)})$, the measure also used
in \cite{dambre_information_2012}.
  }
  \label{input_noise_snr}
\end{figure}

A comparison can be drawn to commonly recommended SNRs in the IEEE 802.11
standard for Wi-Fi communications, ranging from 15db to 25dB. Below this
threshold, wireless communication will quickly become unusable. Additionally, in
\cite{verstraeten_isolated_2005}, experiments with LSM reservoirs containing
1232 leaky integrate integrate-and-fire neurons were performed, adding noise
from the NOISEX database to word recognition tasks. Across noise three types of
noise: speech babble, white-noise, and car interior noise, the error rate
consistently stayed above 80\% with an SNR of 10dB.

In summary, our results indicate a robustness to the presence of noise in the
input stream. This is a promising result when considering physical substrates as
reservoir mediums, providing lower bounds on the ratio of signal power to noise
power that is necessary.

\subsection{Measurement equipment accuracy}

When conducting experiments using physical reservoirs, one will inevitably have
to interact with substrates from software. Whether it be transforming digital
representations of reservoir perturbations to analog signals that cause the
excitation, or the reverse mapping of the analog state of the reservoir into a
digital representation, the accuracy of equipment used for such conversions is
of crucial importance.

Sensor anomalies, noise, and amplification gain may all impact performance, but
as found in the previous subsection, reservoirs prove to be quite robust to the
presence of such noise patterns. Here we thus focus our investigation on the
quantization done in ADC systems.

% (TODO): There are common ADC errors: gain error, linearity error, missing code
% errors, offset errors.

\begin{figure}[H]
  \centering
  \includegraphics[width=2.5in]{img/adc_quantization.png}
  \caption{
    Performance effect of ADC quantization on four reservoirs of different
sizes. $\tanh$ is used as activation function for the experiment, dividing its
range of (1-, 1) into $n$ discrete output bins.
  }
  \label{adc_quantization}
\end{figure}

% (TODO): Change this to bits of quantization? As if it's an actual ADC.

To emulate the behavior of an ADC, we extend our ESN model to allow for
quantization of reservoir output before it is passed to the readout layer. This
quantization effectively divides the range of the nonlinear activation function
of each hidden node into a discrete set of fixed output bins.

Fig. \ref{adc_quantization} shows how quantization affects reservoir
performance. We plot the error of four different reservoir sizes: 50, 100, 200
and 400 hidden nodes.

We see a performance degradation from 3000 to 1000 discernible states, something
of which is more pressing for the bigger reservoirs. As discrete output states
move beneath 1000, the performance quickly deteriorates. However, even with just
10 discrete outputs, reservoirs are still able to replicate the input sequence
to some degree. Furthermore, reservoirs with 400 hidden nodes and just 100
discrete output bins consistently provide the same performance as that of a
reservoir with 50 hidden nodes and no output quantization.

A 12-bit ADC has $2^{12}$, or 4096 output codes, which in accordance with our
experiments would impose no noticeable increase in the prediction error. 16-bit
ADCs, outputting $2^{16}$, or 65536 discrete states, would influence our ESN
simulations even less. In \cite{soriano_delay-based_2015}, a major factor
limiting performance of a nonlinear analog electric circuit implementation was
quantization noise. Low error rates on the Mackey-Glass task were achieved with
8 bits or more in the ADC. This result is also seen in our ESN implementation,
where NRMSE starts leveling out around 8 bits of accuracy, or 256 discrete
states.

These results demonstrate that the reservoir methodology provides an inherent
resilience to output quantization. When designing physical RC systems, this may
be applied in preliminary analysis of the output resolution and representation,
whether it be voltage in electrode arrays or the magnetic polarization of
permalloy nanomagnets.

% (TODO): I don't really like the conclusion here -- could I conclude with
% something more _concrete_, almost like tricks of the trade?

% (TODO): Would higher order NARMAs show different degradations, as they require
% more memory and computation?

\subsection{Partially observable reservoir state}

Consider a physical reservoir system using microelectrode arrays (MEAs) as its
computational substrate, a common approach when using biological, \textit{in
vitro} components \cite{aaser_towards_2017}. The goal of such MEAs is to serve
as an interface that connects biological neuronal activity to electronic
circuitry, and it does so by having an organization of microelectrodes on a
two-dimensional grid. Obtaining neural signals is done only through the
electrode interface by means of a two-way transduction from voltage drop in the
biological environment to a an electric current and vice versa.

When seeding MEAs with solutions containing neuronal cultures, one is by no
means guaranteed a neural network that fits the MEA layout. In fact, with common
grid layouts ranging from 64 to 256 electrodes, each electrode will examine its
surrounding area, not individual cells. Thus, in this section we intend to
provide an insight into the performance effect of having reservoirs that are
only partially observable.

In this section we experiment with the sparsity of $\mathbf{W}_{in}$ and
$\mathbf{W}_{out}$. In both cases, we now generate the connection matrices such
that a wanted density, given as the fraction of connected nodes, is
achieved. Input and output is adjusted separately. Fig. \ref{partial_visibility}
shows the results of our simulation runs. Additionally, we investigate the
effect of fixing the input weight distribution completely, hence only allowing a
constant scaling of the input stream.

When attaching input units to the network by weights, experience has shown that
choosing $\mathbf{W}_{in}$ can be done freely, as long as
$\rho(\mathbf{W}_{res}) < 1$ is satisfied \cite{jaeger_echo_2001}. Experimental
results from adjusting the sparsity of $\mathbf{W}_{in}$ is shown in
Fig. \ref{partial_visibility}a, in which all experiments were done with a
uniform input weight distribution in the interval [-0.5, 0.5].

Across all three reservoir sizes, we see a small increase in the signal
prediction error when increasing input connectivity from an initial density of
0.1. While finding the optimal reservoir perturbance clearly depends on the
memory capacity and computational dynamics required for the task at hand, this
benchmark task gives a clear indication of the influence of input sparsity in
general. The relatively low sensitivity to this parameter suggests that physical
reservoirs may achieve excellent performance even when only perturbing selected
parts of the system. An input stream parameter found to be of greater relevance
is input scaling, i.e. the amplitude of the input signal that the reservoir sees
\cite{alippi_quantification_2009}. The necessary scaling depends on the
nonlinearity needed, as inputs far from 0 drive $\tanh$ neurons towards
saturation. In a physical reservoir design setting, this parameter is often
possible to adjust after-the-fact, as opposed to changing the physical hardware
layout of the system.

Next, a general trend seen when adjusting output connectivity, is a slight,
linear decrease in performance from a completely dense output matrix to a
density of 0.4 (Fig. \ref{partial_visibility}b). Furthermore, for the 50 and 100
node reservoirs, there is a noticeable transition from a linear to an
exponential NRMSE growth.

We may understand these results to indicate that there is a critical point in
the output density space where there the readout layer starts seeing enough of
the reservoir dynamics to reconstruct the input NARMA10 signal. Additionally,
when the output density exceeds this threshold, the performance will continue to
grow linearly with the density, achieving the best performance at full
connectivity.

% (TODO): Also plot output connectivity as amount of output nodes. Perhaps we see
% that the size of the underlying reservoir does not matter at all?

\begin{figure}[H]
  \centering
  \includegraphics[width=2.5in]{img/input_scaling_distrib.png}
  \caption{
    Effect of input scaling on three input weight distributions. With the fixed
distribution every input weight is set to 1.
  }
  \label{input_scaling_distrib}
\end{figure}

In \cite{jensen_computation_2018}, artificial spin ice is explored as a
substrate for physical RC. Here, nanomagnetic assemblies are perturbed, using an
external magnetic field as input source. Thus, every magnet sees exactly the
same input stream.

Fig. \ref{input_scaling_distrib} explores the performance of ESNs with three
different input weight distributions: uniform, Gaussian and fixed. Despite the
performance disparity with increasing input scaling, the best performance of all
three classes hover around the same NRMSE of around 0.25. This demonstration of
scaling differently distributed inputs such that they all achieve acceptable
performance bodes well for physical RC. In particular, we observe a promising
result when considering RC with physical substrates in which every node is
forced to receive the same input.

\begin{figure*}[htbp]
  \centering
  \begin{subfigure}{.3\textwidth}
    \centering
    \includegraphics[width=\linewidth]{img/input_density_all.png}
    \caption{}
  \end{subfigure}
  \begin{subfigure}{.3\textwidth}
    \centering
    \includegraphics[width=\linewidth]{img/output_density_all.png}
    \caption{}
  \end{subfigure}
  \begin{subfigure}{.3\textwidth}
    \centering
    \includegraphics[width=\linewidth]{img/partial_visibility.png}
    \caption{}
  \end{subfigure}
  \caption{
    These figures show the influence of reservoirs that are only partially
visible on the performance for the NARMA10 task. The density is a measurement
for the fraction of elements in the input and output matrices containing
non-zero elements. The heat map was generated using a reservoir size of 200
hidden nodes.
  }
  \label{partial_visibility}
\end{figure*}

% (TODO): Ensure that the subfigure figure isn't pushed _into_ the references.

\subsection{Topology}

Topology section. Probably a literature review, leading into discussions about
future work.

\begin{figure}[H]
  \centering
  \includegraphics[width=2.5in]{img/reservoir_density_distrib.png}
  \caption{
    Reservoir density versus for two weight distributions. Fixed weight will not
work for internal nodes, unless special topology is employed, e.g. ring
(elaborate). This shows that sparse internal weight matrices work fine.
  }
  \label{reservoir_density_distrib}
\end{figure}

Something about internal reservoir density as an introductory exploration.

% (TODO): Reason for reservoir sparsity is in the original Jaeger paper,
% e.g. multiple creating separate networks with rich dynamics.

%%% Local Variables:
%%% mode: latex
%%% TeX-master: "../main"
%%% End:
