This is the background. This should primarily be on Echo State Networks in practice.

% END

% 1. How ESNs work. Why ESNs are chosen to approximate physical systems; what can
% it tell us about such physical limitations? Cite practical guide to ESNs by Luko
% and Adaptive Nonlinear System Identification.

% 2. Time series, specifically NARMA and NARMA10. Also write about practical
% applications of RC range. Why are ESNs not useless? Some example applications in
% Verstraeten's thesis and Reservoir Computing Trends by Jaeger. Also relate this
% to Dambre paper on the fundamental information processing capabilities, it's a
% fundamental question.

% 3. Ridge regression, pseudo inverse, how it's done. Should perhaps be part of
% ESN background.

% Evaluation of ESN performance on NARMA10 has been done in: An experimental
% unification of reservoir computing methods, Minimum Complexity Echo State
% Networks, Adaptive Nonlinear System Identification with Echo State Networks.

% Introduce the problem statements individually for each limitation. ''Natural
% computational systems must have specific topologies, and the uniform random
% connectivity is not appropriate.'' Topology discussion, Manevitz et. al.

%%% Local Variables:
%%% mode: latex
%%% TeX-master: "../main"
%%% End:
